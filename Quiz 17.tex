% Type of the document
\documentclass{beamer}

% elementary packages:
\usepackage{graphicx}
\usepackage[latin1]{inputenc}
\usepackage[T1]{fontenc}
\usepackage[english]{babel}
\usepackage{listings}
\usepackage{xcolor}
\usepackage{eso-pic}
\usepackage{mathrsfs}
\usepackage{url}
\usepackage{amssymb}
\usepackage{amsmath}
\usepackage{multirow}
\usepackage{hyperref}
\usepackage{booktabs}

% additional packages
\usepackage{bbm}

% packages supplied with ise-beamer:
\usepackage{cooltooltips}
\usepackage{colordef}
\usepackage{beamerdefs}
\usepackage{lvblisting}

% Change the pictures here:
% logobig and logosmall are the internal names for the pictures: do not modify them. 
% Pictures must be supplied as JPEG, PNG or, to be preferred, PDF
\pgfdeclareimage[height=2cm]{logobig}{hulogo}
% Supply the correct logo for your class and change the file name to "logo". The logo will appear in the lower
% right corner:
\pgfdeclareimage[height=0.7cm]{logosmall}{Figures/LOB_Logo}

% Title page outline:
% use this number to modify the scaling of the headline on title page
\renewcommand{\titlescale}{1.0}
% the title page has two columns, the following two values determine the percentage each one should get
\renewcommand{\titlescale}{1.0}
\renewcommand{\leftcol}{0.6}

% Define the title.Don't forget to insert an abbreviation instead 
% of "title for footer". It will appear in the lower left corner:
\title{Selected Topics of Mathematical Statistics: Quiz 17}
% Define the authors:
\authora{Xiang Gao} % a-c
\authorb{}
\authorc{}

% Define any internet addresses, if you want to display them on the title page:
\def\linka{http://lvb.wiwi.hu-berlin.de}
\def\linkb{}
\def\linkc{}

% Define the institute:
\institute{Ladislaus von Bortkiewicz Chair of Statistics \\
Humboldt--Universit{\"a}t zu Berlin \\}

% Comment the following command, if you don't want, that the pdf file starts in full screen mode:
\hypersetup{pdfpagemode=FullScreen}

%Start of the document
\begin{document}

% create the title slide, layout controlled in beamerdefs.sty and the foregoing specifications
\frame[plain]{
\titlepage
}

% The titles of the different sections of you talk, can be included via the \section command. The title will be displayed in the upper left corner. To indicate a new section, repeat the \section command with, of course, another section title
%%%%%%%%%%%%%%%%%%%%%%%%%%%%%%%%%%%%%%%%%%%%%%%%%%%%%%%%%%%%%%%%%%%%%%%%%%%%%%%%%%%%%%%%%%%%%%%%%%%%%%%%%%%%%%%%%%%%%%%%
\section{Solutions to quizzes}
%%%%%%%%%%%%%%%%%%%%%%%%%%%%%%%%%%%%%%%%%%%%%%%%%%%%%%%%%%%%%%%%%%%%%%%%%%%%%%%%%%%%%%%%%%%%%%%%%%%%%%%%%%%%%%%%%%%%%%%%

% (A numbering of the slides can be useful for corrections, especially if you are
% dealing with large tex-files)
%%%%%%%%%%%%%%%%%%%%%%%%%%%%%%%%%%%%%%%%%%%%%%%%%%%%%%%%%%%%%%%%%%%%%%%%%%%%%%%%%%%%%%%%%%%%%%%%%%%%%%%%%%%%%%%%%%%%%%%%


%%%%%%%%%%%%%%%%%%%%%%%%%%%%%%%%%%%%%%%%%%%%%%%%%%%%%%%%%%%%%%%%%%%%%%%%%%%%%%%%%%%%%%%%%%%%%%%%%%%%%%%%%%%%%%%%%%%%%%%%
\frame{
\frametitle{Quiz 17: Question}

\textbf{Proof of Theorem 37}\\
For each  
\begin{math}
	-\infty<x<\infty
\end{math}
,
\begin{align*}
	\sqrt{n}\lbrace F_n(x)-F(x)\rbrace\overset{\mathcal{L}}{\to}
		N[0,F(x)\lbrace 1-F(x)\rbrace]
\end{align*}

where
\begin{math}
	F_n(x)=\frac{1}{n} \sum_{i=1}^n \textbf{I} (X_i \leqslant x)
\end{math}
is the empirical distribution function of an $iid$ sequence $\lbrace X_i\rbrace$ with cdf $F$.
}
%%%%%%%%%%%%%%%%%%%%%%%%%%%%%%%%%%%%%%%%%%%%%%%%%%%%%%%%%%%%%%%%%%%%%%%%%%%%%%%%%%%%%%%%%%%%%%%%%%%%%%%%%%%%%%%%%%%%%%%%
\frame{
\frametitle{Quiz 17: Solution}
\begin{itemize}
\item Set  
\begin{math}
	Yi:= \textbf{I} (X_i \leqslant x)
\end{math}
,
\begin{math}
	i \in \mathbb{N}
\end{math}
, so that
\begin{math}
	\bar{Y_n}
		=\frac{1}{n}\sum_{i=1}^n Y_i 
		=\frac{1}{n}\sum_{i=1}^n \textbf{I}(X_i \leqslant x)=F_n(x)
\end{math}
.\\
\begin{math}
	\lbrace Y_i \rbrace 
\end{math}
is therefore also an $iid$ sequence, since $\lbrace X_i \rbrace$ is $iid$ and indicate function is measurable, and
\end{itemize}
\begin{align*}
	E[Y_i]
		=E[\textbf{I}(X_i \leqslant x)]
		=P(X_i \leqslant x)=F(x)
\end{align*}
\begin{align*}
	Var(Y_i)
		&=E[\underbrace{(\textbf{I}(X_i \leqslant x))^2}_{\textbf{I}(X_i \leqslant x)}]-(E[\textbf{I}(X_i \leqslant x)])^2\\
		&=F(x)-(F(x))^2=F(x)(1-F(x))
\end{align*}
}

%%%%%%%%%%%%%%%%%%%%%%%%%%%%%%%%%%%%%%%%%%%%%%%%%%%%%%%%%%%%%%%%%%%%%%%%%%%%%%%%%%%%%%%%%%%%%%%%%%%%%%%%%%%%%%%%%%%%%%%%
\frame{
\frametitle{Quiz 17: Solution}
\begin{enumerate}[(i)]
\item If 
\begin{math}
	F(x)\in (0, 1)
\end{math}
, we can get from Theorem 31 that
\begin{align*}
	\sqrt{n} (\bar{Y_n}-F(x))
		= \sqrt{n}(F_n(x)-F(x)) \overset{\mathcal{L}}{\to} N(0,F(x)(1-F(x))
\end{align*}

\end{enumerate}
}
%%%%%%%%%%%%%%%%%%%%%%%%%%%%%%%%%%%%%%%%%%%%%%%%%%%%%%%%%%%%%%%%%%%%%%%%%%%%%%%%%%%%%%%%%%%%%%%%%%%%%%%%%%%%%%%%%%%%%%%%
\frame{
\frametitle{Quiz 17: Solution}
\begin{enumerate}[(ii)]
\item If 
\begin{math}
	F(x)=P(X_i \leqslant x)=0
\end{math}
, so that 
\begin{math}
	Y_i=\textbf{I} (X_i \leqslant x)=0
\end{math}
 a.s. That means %$Y_i \to 0$ a.s., and that 
\begin{align*}
	F_n(x)=\frac{1}{n}\sum_{i=1}^n \textbf{I}(X_i \leqslant x)=0 \quad a.s.
\end{align*}
So that 
\begin{align*}
	\sqrt{n}(F_n(x)-F(x)) \overset{a.s.}{\to} 0
\end{align*}
and thus
\begin{align*}
	\sqrt{n}(F_n(x)-F(x)) \overset{\mathcal{L}}{\to} 0
\end{align*}
\end{enumerate}
}
%%%%%%%%%%%%%%%%%%%%%%%%%%%%%%%%%%%%%%%%%%%%%%%%%%%%%%%%%%%%%%%%%%%%%%%%%%%%%%%%%%%%%%%%%%%%%%%%%%%%%%%%%%%%%%%%%%%%%%%%
\frame{
\frametitle{Quiz 17: Solution}
\begin{enumerate}[(iii)]
\item If 
\begin{math}
	F(x)=P(X_i \leqslant x)=1
\end{math}
, so that 
\begin{math}
	Y_i=\textbf{I} (X_i \leqslant x)=1
\end{math}
a.s. That means %$Y_i \to 1$ a.s., and that 
\begin{align*}
	F_n(x)=\frac{1}{n}\sum_{i=1}^n \textbf{I}(X_i \leqslant x)=\frac{n}{n}=1 \quad a.s.
\end{align*}
So that 
\begin{align*}
	\sqrt{n}(F_n(x)-F(x)) \overset{a.s.}{\to} 0
\end{align*}
and thus
\begin{align*}
	\sqrt{n}(F_n(x)-F(x)) \overset{\mathcal{L}}{\to} 0
\end{align*}
\end{enumerate}
}

% Define the end of the document:
\end{document}
